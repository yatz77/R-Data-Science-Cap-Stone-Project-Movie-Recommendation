% Options for packages loaded elsewhere
\PassOptionsToPackage{unicode}{hyperref}
\PassOptionsToPackage{hyphens}{url}
%
\documentclass[
]{article}
\usepackage{lmodern}
\usepackage{amssymb,amsmath}
\usepackage{ifxetex,ifluatex}
\ifnum 0\ifxetex 1\fi\ifluatex 1\fi=0 % if pdftex
  \usepackage[T1]{fontenc}
  \usepackage[utf8]{inputenc}
  \usepackage{textcomp} % provide euro and other symbols
\else % if luatex or xetex
  \usepackage{unicode-math}
  \defaultfontfeatures{Scale=MatchLowercase}
  \defaultfontfeatures[\rmfamily]{Ligatures=TeX,Scale=1}
\fi
% Use upquote if available, for straight quotes in verbatim environments
\IfFileExists{upquote.sty}{\usepackage{upquote}}{}
\IfFileExists{microtype.sty}{% use microtype if available
  \usepackage[]{microtype}
  \UseMicrotypeSet[protrusion]{basicmath} % disable protrusion for tt fonts
}{}
\makeatletter
\@ifundefined{KOMAClassName}{% if non-KOMA class
  \IfFileExists{parskip.sty}{%
    \usepackage{parskip}
  }{% else
    \setlength{\parindent}{0pt}
    \setlength{\parskip}{6pt plus 2pt minus 1pt}}
}{% if KOMA class
  \KOMAoptions{parskip=half}}
\makeatother
\usepackage{xcolor}
\IfFileExists{xurl.sty}{\usepackage{xurl}}{} % add URL line breaks if available
\IfFileExists{bookmark.sty}{\usepackage{bookmark}}{\usepackage{hyperref}}
\hypersetup{
  pdftitle={HarvardX PH125.9xData Science: Capstone: Movielens recommendation system},
  pdfauthor={Yannick Hermans},
  hidelinks,
  pdfcreator={LaTeX via pandoc}}
\urlstyle{same} % disable monospaced font for URLs
\usepackage[margin=1in]{geometry}
\usepackage{color}
\usepackage{fancyvrb}
\newcommand{\VerbBar}{|}
\newcommand{\VERB}{\Verb[commandchars=\\\{\}]}
\DefineVerbatimEnvironment{Highlighting}{Verbatim}{commandchars=\\\{\}}
% Add ',fontsize=\small' for more characters per line
\usepackage{framed}
\definecolor{shadecolor}{RGB}{248,248,248}
\newenvironment{Shaded}{\begin{snugshade}}{\end{snugshade}}
\newcommand{\AlertTok}[1]{\textcolor[rgb]{0.94,0.16,0.16}{#1}}
\newcommand{\AnnotationTok}[1]{\textcolor[rgb]{0.56,0.35,0.01}{\textbf{\textit{#1}}}}
\newcommand{\AttributeTok}[1]{\textcolor[rgb]{0.77,0.63,0.00}{#1}}
\newcommand{\BaseNTok}[1]{\textcolor[rgb]{0.00,0.00,0.81}{#1}}
\newcommand{\BuiltInTok}[1]{#1}
\newcommand{\CharTok}[1]{\textcolor[rgb]{0.31,0.60,0.02}{#1}}
\newcommand{\CommentTok}[1]{\textcolor[rgb]{0.56,0.35,0.01}{\textit{#1}}}
\newcommand{\CommentVarTok}[1]{\textcolor[rgb]{0.56,0.35,0.01}{\textbf{\textit{#1}}}}
\newcommand{\ConstantTok}[1]{\textcolor[rgb]{0.00,0.00,0.00}{#1}}
\newcommand{\ControlFlowTok}[1]{\textcolor[rgb]{0.13,0.29,0.53}{\textbf{#1}}}
\newcommand{\DataTypeTok}[1]{\textcolor[rgb]{0.13,0.29,0.53}{#1}}
\newcommand{\DecValTok}[1]{\textcolor[rgb]{0.00,0.00,0.81}{#1}}
\newcommand{\DocumentationTok}[1]{\textcolor[rgb]{0.56,0.35,0.01}{\textbf{\textit{#1}}}}
\newcommand{\ErrorTok}[1]{\textcolor[rgb]{0.64,0.00,0.00}{\textbf{#1}}}
\newcommand{\ExtensionTok}[1]{#1}
\newcommand{\FloatTok}[1]{\textcolor[rgb]{0.00,0.00,0.81}{#1}}
\newcommand{\FunctionTok}[1]{\textcolor[rgb]{0.00,0.00,0.00}{#1}}
\newcommand{\ImportTok}[1]{#1}
\newcommand{\InformationTok}[1]{\textcolor[rgb]{0.56,0.35,0.01}{\textbf{\textit{#1}}}}
\newcommand{\KeywordTok}[1]{\textcolor[rgb]{0.13,0.29,0.53}{\textbf{#1}}}
\newcommand{\NormalTok}[1]{#1}
\newcommand{\OperatorTok}[1]{\textcolor[rgb]{0.81,0.36,0.00}{\textbf{#1}}}
\newcommand{\OtherTok}[1]{\textcolor[rgb]{0.56,0.35,0.01}{#1}}
\newcommand{\PreprocessorTok}[1]{\textcolor[rgb]{0.56,0.35,0.01}{\textit{#1}}}
\newcommand{\RegionMarkerTok}[1]{#1}
\newcommand{\SpecialCharTok}[1]{\textcolor[rgb]{0.00,0.00,0.00}{#1}}
\newcommand{\SpecialStringTok}[1]{\textcolor[rgb]{0.31,0.60,0.02}{#1}}
\newcommand{\StringTok}[1]{\textcolor[rgb]{0.31,0.60,0.02}{#1}}
\newcommand{\VariableTok}[1]{\textcolor[rgb]{0.00,0.00,0.00}{#1}}
\newcommand{\VerbatimStringTok}[1]{\textcolor[rgb]{0.31,0.60,0.02}{#1}}
\newcommand{\WarningTok}[1]{\textcolor[rgb]{0.56,0.35,0.01}{\textbf{\textit{#1}}}}
\usepackage{longtable,booktabs}
% Correct order of tables after \paragraph or \subparagraph
\usepackage{etoolbox}
\makeatletter
\patchcmd\longtable{\par}{\if@noskipsec\mbox{}\fi\par}{}{}
\makeatother
% Allow footnotes in longtable head/foot
\IfFileExists{footnotehyper.sty}{\usepackage{footnotehyper}}{\usepackage{footnote}}
\makesavenoteenv{longtable}
\usepackage{graphicx,grffile}
\makeatletter
\def\maxwidth{\ifdim\Gin@nat@width>\linewidth\linewidth\else\Gin@nat@width\fi}
\def\maxheight{\ifdim\Gin@nat@height>\textheight\textheight\else\Gin@nat@height\fi}
\makeatother
% Scale images if necessary, so that they will not overflow the page
% margins by default, and it is still possible to overwrite the defaults
% using explicit options in \includegraphics[width, height, ...]{}
\setkeys{Gin}{width=\maxwidth,height=\maxheight,keepaspectratio}
% Set default figure placement to htbp
\makeatletter
\def\fps@figure{htbp}
\makeatother
\setlength{\emergencystretch}{3em} % prevent overfull lines
\providecommand{\tightlist}{%
  \setlength{\itemsep}{0pt}\setlength{\parskip}{0pt}}
\setcounter{secnumdepth}{-\maxdimen} % remove section numbering

\title{HarvardX PH125.9xData Science: Capstone: Movielens recommendation system}
\author{Yannick Hermans}
\date{1-10-2020}

\begin{document}
\maketitle

\hypertarget{introduction}{%
\subsection{Introduction}\label{introduction}}

In 2006 Netflix launched a competition where participants were asked to
optimize Netflix's movie recommendation system with at least 10 \%. The
winners were promised a prize worth \$1M. The challenge was to predict
user ratings as accurately as possible based on a dataset containing
previous user ratings. There was no additional knowledge on the users or
movies. In 2009 BellKor's Pragmatic Chaos team was able to best
Netflix's movie recommendation algorithm by 10.06\%

In this capstone project, which forms the conclusion to the HarvardX
PH125.9xData Science course, the goal is to build towards the
recommendation algorithm that the winners of the Netflix challenge
developed. This project will represent the knowledge I gained throughout
the edX data science course on machine learning algorithms, data
processing and data exploration. A reduced version of the open source
MovieLens data set with 10M ratings, from 10,000 users on 72,000 movies,
has been used for this data science project.

This report contains the data processing steps, data exploratory
analysis and a few machine learning algorithms.

\hypertarget{data-preprocessing}{%
\subsection{Data preprocessing}\label{data-preprocessing}}

First the necessary packages are installed and loaded. The 10M MovieLens
dataset is loaded and separated into a training (``edx'') and a
validation ``validation'' set. The code for these preprocessing steps is
based on the already provided code in the edx capstone project module.
\url{https://courses.edx.org/courses/course-v1:HarvardX+PH125.9x+2T2018/courseware/dd9a048b16ca477a8f0aaf1d888f0734/e8800e37aa444297a3a2f35bf84ce452/?child=first}

\begin{Shaded}
\begin{Highlighting}[]
\CommentTok{#########################################################}
\CommentTok{# Create edx set, validation set (final hold_out test set)}
\CommentTok{#########################################################}

\CommentTok{# Note: this process could take a couple of minutes}

\ControlFlowTok{if}\NormalTok{(}\OperatorTok{!}\KeywordTok{require}\NormalTok{(tidyverse)) }\KeywordTok{install.packages}\NormalTok{(}\StringTok{"tidyverse"}\NormalTok{, }\DataTypeTok{repos =} \StringTok{"http://cran.us.r-project.org"}\NormalTok{)}
\ControlFlowTok{if}\NormalTok{(}\OperatorTok{!}\KeywordTok{require}\NormalTok{(caret)) }\KeywordTok{install.packages}\NormalTok{(}\StringTok{"caret"}\NormalTok{, }\DataTypeTok{repos =} \StringTok{"http://cran.us.r-project.org"}\NormalTok{)}
\ControlFlowTok{if}\NormalTok{(}\OperatorTok{!}\KeywordTok{require}\NormalTok{(data.table)) }\KeywordTok{install.packages}\NormalTok{(}\StringTok{"data.table"}\NormalTok{, }\DataTypeTok{repos =} \StringTok{"http://cran.us.r-project.org"}\NormalTok{)}

\KeywordTok{library}\NormalTok{(tidyverse)}
\KeywordTok{library}\NormalTok{(caret)}
\KeywordTok{library}\NormalTok{(data.table)}

\CommentTok{# MovieLens 10M dataset:}
\CommentTok{# https://grouplens.org/datasets/movielens/10m/}
\CommentTok{# http://files.grouplens.org/datasets/movielens/ml-10m.zip}

\NormalTok{dl <-}\StringTok{ }\KeywordTok{tempfile}\NormalTok{()}
\KeywordTok{download.file}\NormalTok{(}\StringTok{"http://files.grouplens.org/datasets/movielens/ml-10m.zip"}\NormalTok{, dl)}
\NormalTok{ratings <-}\StringTok{ }\KeywordTok{fread}\NormalTok{(}\DataTypeTok{text =} \KeywordTok{gsub}\NormalTok{(}\StringTok{"::"}\NormalTok{, }\StringTok{"}\CharTok{\textbackslash{}t}\StringTok{"}\NormalTok{, }\KeywordTok{readLines}\NormalTok{(}\KeywordTok{unzip}\NormalTok{(dl, }\StringTok{"ml-10M100K/ratings.dat"}\NormalTok{))),}
                 \DataTypeTok{col.names =} \KeywordTok{c}\NormalTok{(}\StringTok{"userId"}\NormalTok{, }\StringTok{"movieId"}\NormalTok{, }\StringTok{"rating"}\NormalTok{, }\StringTok{"timestamp"}\NormalTok{))}

\NormalTok{movies <-}\StringTok{ }\KeywordTok{str_split_fixed}\NormalTok{(}\KeywordTok{readLines}\NormalTok{(}\KeywordTok{unzip}\NormalTok{(dl, }\StringTok{"ml-10M100K/movies.dat"}\NormalTok{)), }\StringTok{"}\CharTok{\textbackslash{}\textbackslash{}}\StringTok{::"}\NormalTok{, }\DecValTok{3}\NormalTok{)}
\KeywordTok{colnames}\NormalTok{(movies) <-}\StringTok{ }\KeywordTok{c}\NormalTok{(}\StringTok{"movieId"}\NormalTok{, }\StringTok{"title"}\NormalTok{, }\StringTok{"genres"}\NormalTok{)}

\NormalTok{movies <-}\StringTok{ }\KeywordTok{as.data.frame}\NormalTok{(movies) }\OperatorTok\StringTok{ }\KeywordTok{mutate}\NormalTok{(}\DataTypeTok{movieId =} \KeywordTok{as.numeric}\NormalTok{(movieId),}
                                           \DataTypeTok{title =} \KeywordTok{as.character}\NormalTok{(title),}
                                           \DataTypeTok{genres =} \KeywordTok{as.character}\NormalTok{(genres))}

\NormalTok{movielens <-}\StringTok{ }\KeywordTok{left_join}\NormalTok{(ratings, movies, }\DataTypeTok{by =} \StringTok{"movieId"}\NormalTok{)}

\CommentTok{# Validation set will contain 10% of the MovieLens data set}
\KeywordTok{set.seed}\NormalTok{(}\DecValTok{1}\NormalTok{, }\DataTypeTok{sample.kind =} \StringTok{"Rounding"}\NormalTok{)}
\NormalTok{test_index <-}\StringTok{ }\KeywordTok{createDataPartition}\NormalTok{(}\DataTypeTok{y =}\NormalTok{ movielens}\OperatorTok{$}\NormalTok{rating, }\DataTypeTok{times =} \DecValTok{1}\NormalTok{, }\DataTypeTok{p =} \FloatTok{0.1}\NormalTok{, }\DataTypeTok{list =} \OtherTok{FALSE}\NormalTok{)}
\NormalTok{edx <-}\StringTok{ }\NormalTok{movielens[}\OperatorTok{-}\NormalTok{test_index,]}
\NormalTok{temp <-}\StringTok{ }\NormalTok{movielens[test_index,]}
\CommentTok{# Make sure userId and movieId in validation set are also in edx set}
\NormalTok{validation <-}\StringTok{ }\NormalTok{temp }\OperatorTok
\StringTok{  }\KeywordTok{semi_join}\NormalTok{(edx, }\DataTypeTok{by =} \StringTok{"movieId"}\NormalTok{) }\OperatorTok
\StringTok{  }\KeywordTok{semi_join}\NormalTok{(edx, }\DataTypeTok{by =} \StringTok{"userId"}\NormalTok{)}
\CommentTok{# Add rows removed from validation set back into edx set}
\NormalTok{removed <-}\StringTok{ }\KeywordTok{anti_join}\NormalTok{(temp, validation)}
\NormalTok{edx <-}\StringTok{ }\KeywordTok{rbind}\NormalTok{(edx, removed)}

\KeywordTok{rm}\NormalTok{(dl, ratings, movies, test_index, temp, movielens, removed)}
\end{Highlighting}
\end{Shaded}

The validation set is used to determine the accuracy of the final movie
recommendation algorithm. The other models are tested by splitting up
the edx set into a training (edx\_training\_set) and a test set
(edx\_test\_set).

\begin{Shaded}
\begin{Highlighting}[]
\CommentTok{# Split edx data in training (80%) and test set (20%) }
\KeywordTok{set.seed}\NormalTok{(}\DecValTok{1992}\NormalTok{, }\DataTypeTok{sample.kind =} \StringTok{"Rounding"}\NormalTok{)}
\NormalTok{edx_test_index <-}\StringTok{ }\KeywordTok{createDataPartition}\NormalTok{(}\DataTypeTok{y =}\NormalTok{ edx}\OperatorTok{$}\NormalTok{rating, }\DataTypeTok{times =} \DecValTok{1}\NormalTok{,}
                                  \DataTypeTok{p =} \FloatTok{0.2}\NormalTok{, }\DataTypeTok{list =} \OtherTok{FALSE}\NormalTok{)}
\NormalTok{edx_train_set <-}\StringTok{ }\NormalTok{edx[}\OperatorTok{-}\NormalTok{edx_test_index,]}
\NormalTok{edx_test_set <-}\StringTok{ }\NormalTok{edx[edx_test_index,]}

\NormalTok{edx_test_set <-}\StringTok{ }\NormalTok{edx_test_set }\OperatorTok
\StringTok{  }\KeywordTok{semi_join}\NormalTok{(edx_train_set, }\DataTypeTok{by =} \StringTok{"movieId"}\NormalTok{) }\OperatorTok
\StringTok{  }\KeywordTok{semi_join}\NormalTok{(edx_train_set, }\DataTypeTok{by =} \StringTok{"userId"}\NormalTok{)}
\end{Highlighting}
\end{Shaded}

In addition, certain libraries for the data exploration and machine
learning development are loaded

\begin{Shaded}
\begin{Highlighting}[]
\KeywordTok{library}\NormalTok{(lubridate)}
\KeywordTok{library}\NormalTok{(ggplot2)}
\KeywordTok{library}\NormalTok{(recosystem)}
\KeywordTok{library}\NormalTok{(knitr)}
\KeywordTok{library}\NormalTok{(gridExtra)}
\end{Highlighting}
\end{Shaded}

\hypertarget{data-exploratory-analysis}{%
\subsection{Data exploratory analysis}\label{data-exploratory-analysis}}

\hypertarget{overview}{%
\subsubsection{Overview}\label{overview}}

First an overview of the data is given.

\begin{Shaded}
\begin{Highlighting}[]
\KeywordTok{kable}\NormalTok{(}\KeywordTok{head}\NormalTok{(edx))}
\end{Highlighting}
\end{Shaded}

\begin{longtable}[]{@{}lrrrrll@{}}
\toprule
& userId & movieId & rating & timestamp & title & genres\tabularnewline
\midrule
\endhead
1 & 1 & 122 & 5 & 838985046 & Boomerang (1992) &
Comedy\textbar Romance\tabularnewline
2 & 1 & 185 & 5 & 838983525 & Net, The (1995) &
Action\textbar Crime\textbar Thriller\tabularnewline
4 & 1 & 292 & 5 & 838983421 & Outbreak (1995) &
Action\textbar Drama\textbar Sci-Fi\textbar Thriller\tabularnewline
5 & 1 & 316 & 5 & 838983392 & Stargate (1994) &
Action\textbar Adventure\textbar Sci-Fi\tabularnewline
6 & 1 & 329 & 5 & 838983392 & Star Trek: Generations (1994) &
Action\textbar Adventure\textbar Drama\textbar Sci-Fi\tabularnewline
7 & 1 & 355 & 5 & 838984474 & Flintstones, The (1994) &
Children\textbar Comedy\textbar Fantasy\tabularnewline
\bottomrule
\end{longtable}

\begin{Shaded}
\begin{Highlighting}[]
\CommentTok{# the number of observations in the Movielens data set}
\KeywordTok{length}\NormalTok{(edx}\OperatorTok{$}\NormalTok{rating)}\OperatorTok{+}\KeywordTok{length}\NormalTok{(validation}\OperatorTok{$}\NormalTok{rating)}
\end{Highlighting}
\end{Shaded}

{[}1{]} 10000054

The data contains 10000054 observations and is already in a tidy format,
which allows for straightforward data exploration in r. In addition to a
rating, additional information can be found in the dataset such as a
userID, a movieID, the time at which the rating was given and the genres
to which the rated movie belongs. Before analyzing how these variables
relate to the rating, an overview of the rating distribution is shown
below.

\includegraphics{Movielens_project_files/figure-latex/User-rating-distribution-1.pdf}

The user ratings' distribution demonstrates that users mainly give a
score of 3.0 or 4.0. In addition it is clear that users are less likely
to give half point scores than whole point scores. In average users gave
a score of 3.5 which is indicated by the blue dashed line.

For the development of a recommendation algorithm it would be
informative to know how the additional variables in the 10M Movielens
dataset relate to the ratings. This will be explored in the following.

\hypertarget{relation-movieidmoviename-to-user-rating}{%
\subsubsection{Relation MovieId(Moviename) to user
rating}\label{relation-movieidmoviename-to-user-rating}}

\includegraphics{Movielens_project_files/figure-latex/boxplot-movies-1.pdf}

The boxplot above shows the variability of the rating averages for each
movie in the edx dataset. On first sight it seems that the movieId has a
strong influence on the rating, but let's have a look at the 10 best and
the 10 worst rated movies.

\begin{longtable}[]{@{}lrr@{}}
\toprule
title & movieId & mean\_rating\tabularnewline
\midrule
\endhead
Blue Light, The (Das Blaue Licht) (1932) & 64275 & 5.00\tabularnewline
Fighting Elegy (Kenka erejii) (1966) & 51209 & 5.00\tabularnewline
Hellhounds on My Trail (1999) & 3226 & 5.00\tabularnewline
Satan's Tango (Sátántangó) (1994) & 33264 & 5.00\tabularnewline
Shadows of Forgotten Ancestors (1964) & 42783 & 5.00\tabularnewline
Sun Alley (Sonnenallee) (1999) & 53355 & 5.00\tabularnewline
Constantine's Sword (2007) & 65001 & 4.75\tabularnewline
Human Condition II, The (Ningen no joken II) (1959) & 26048 &
4.75\tabularnewline
Human Condition III, The (Ningen no joken III) (1961) & 26073 &
4.75\tabularnewline
Who's Singin' Over There? (a.k.a. Who Sings Over There) (Ko to tamo
peva) (1980) & 5194 & 4.75\tabularnewline
\bottomrule
\end{longtable}

\begin{longtable}[]{@{}lrr@{}}
\toprule
title & movieId & mean\_rating\tabularnewline
\midrule
\endhead
Accused (Anklaget) (2005) & 61768 & 0.5000000\tabularnewline
Besotted (2001) & 5805 & 0.5000000\tabularnewline
Confessions of a Superhero (2007) & 63828 & 0.5000000\tabularnewline
Hi-Line, The (1999) & 8394 & 0.5000000\tabularnewline
War of the Worlds 2: The Next Wave (2008) & 64999 &
0.5000000\tabularnewline
SuperBabies: Baby Geniuses 2 (2004) & 8859 & 0.7946429\tabularnewline
Hip Hop Witch, Da (2000) & 7282 & 0.8214286\tabularnewline
Disaster Movie (2008) & 61348 & 0.8593750\tabularnewline
From Justin to Kelly (2003) & 6483 & 0.9020101\tabularnewline
Criminals (1996) & 604 & 1.0000000\tabularnewline
\bottomrule
\end{longtable}

The top 10 and bottom 10 movies are,, however, rather obscure.
Furthermore, many of these movies have scores which are multiples of
0.5, indicating that only few people have rated those movies. Thus, the
ratings already given for these movies will not be very predictive for
how other users may rate these movies.

\includegraphics{Movielens_project_files/figure-latex/histogram-movies-1.pdf}

Indeed, the histogram above shows that only few movies have more than
1000 ratings and many have less than 100 reviews.

\includegraphics{Movielens_project_files/figure-latex/mean-rating-vs-n-reviews-per-movie-1.pdf}

How the average ratings per movie depend on the number of reviews is
even better visualized in the scatter plot above. It shows that movies
which have been rated often, generally have a higher mean rating.
Additionally, the lower the number of ratings, the stronger the
variability of the rating. After a certain number of ratings, the
average rating of a movie, is likely a strong predictor for what an
unknown user would rate this movie.

\hypertarget{relation-useridusers-to-user-rating}{%
\subsubsection{Relation UserId(Users) to user
rating}\label{relation-useridusers-to-user-rating}}

Now the relation between the userId and the rating will be shown.

\includegraphics{Movielens_project_files/figure-latex/boxplot-users-1.pdf}

The boxplot above resembles the one for the movie - average rating
relationship. The variability here is slightly lower though, since the
interquartile range is narrower. However, there are still many outliers
in the lower and higher rating range. These outliers can be due to users
rating every movie highly/lowly or it could be users which do not have
rated many movies yet.

\includegraphics{Movielens_project_files/figure-latex/histogram-users-1.pdf}

The histogram for the number of ratings per user indeed shows that most
users have rated less than 100 movies. Their rating will be off course
less predictive than for users who predicted more than 1000 movies.

\includegraphics{Movielens_project_files/figure-latex/mean-rating-vs-n-reviews-per-user-1.pdf}

The relationship between a user's average rating and the number of
movies the user has rated is shown above. The variability in the average
rating indeed goes down with the number of reviews a user has posted. It
is, however, clear that some users give generally higher scores than
others, making the UserId a good predictor for the rating.

\hypertarget{relation-genres-to-user-rating}{%
\subsubsection{Relation Genres to user
rating}\label{relation-genres-to-user-rating}}

Another variable which is contained in the 10M movielens dataset are the
genres to which a movie belongs. The genre information is, however,
compacted into one string per movie. To allow for further investigation
each individual genre has to be extracted from the string

\begin{Shaded}
\begin{Highlighting}[]
\CommentTok{# number of ratings for each genre}
\NormalTok{edx_split_genres  <-}\StringTok{ }\NormalTok{edx  }\OperatorTok\StringTok{ }\KeywordTok{separate_rows}\NormalTok{(genres, }\DataTypeTok{sep =} \StringTok{"}\CharTok{\textbackslash{}\textbackslash{}}\StringTok{|"}\NormalTok{)}
\KeywordTok{kable}\NormalTok{(}\KeywordTok{head}\NormalTok{(edx_split_genres))}
\end{Highlighting}
\end{Shaded}

\begin{longtable}[]{@{}rrrrll@{}}
\toprule
userId & movieId & rating & timestamp & title & genres\tabularnewline
\midrule
\endhead
1 & 122 & 5 & 838985046 & Boomerang (1992) & Comedy\tabularnewline
1 & 122 & 5 & 838985046 & Boomerang (1992) & Romance\tabularnewline
1 & 185 & 5 & 838983525 & Net, The (1995) & Action\tabularnewline
1 & 185 & 5 & 838983525 & Net, The (1995) & Crime\tabularnewline
1 & 185 & 5 & 838983525 & Net, The (1995) & Thriller\tabularnewline
1 & 292 & 5 & 838983421 & Outbreak (1995) & Action\tabularnewline
\bottomrule
\end{longtable}

Now the rows only contain one genre.

\includegraphics{Movielens_project_files/figure-latex/n-ratings-for-each-genre-1.pdf}

From the above plot one can see that particular genres were rated more
often than others. This can be due to some genres being more popular
than others or due to more movies having been made for that particular
genre.

\includegraphics{Movielens_project_files/figure-latex/n-movies-for-each-genre-1.pdf}

The barplot above depicting the number of movies for each genre is very
similar to the barplot for the number of ratings for each genre. This
demonstrates that the number of reviews for a genre depends in large on
how many movies have been made for that genre. To see which genres users
like to rate more often, the average number of reviews per movie for
each genre is shown below.

\includegraphics{Movielens_project_files/figure-latex/n-ratings-per-movie-for-each-genre-1.pdf}

It can be seen that users like to rate Adventure and Sci-Fi movies more
often than documentaries and IMAX movies. This could be an indication
that some genres are more popular than others.

\includegraphics{Movielens_project_files/figure-latex/confidence-intervals-for-each-genre-1.pdf}

However, the plot above showing the average rating and the 95\%
confidence interval per genre is quite different from the previous bar
plot. Thus, movies which users like to rate do not automatically yield
higher average ratings. In addition, the average ratings per genre do
not differ strongly from the overall rating average (blue dashed line).
Thus, the genre to which a movie belongs does not influence the rating
much. This is also clearly visualized in the boxplot below.

\includegraphics{Movielens_project_files/figure-latex/boxplot-genres-1.pdf}

\hypertarget{relation-time-of-release-and-age-at-rating-to-user-rating}{%
\subsubsection{Relation time of release and age at rating to user
rating}\label{relation-time-of-release-and-age-at-rating-to-user-rating}}

The last features in the dataset are the date of release of the movie
(mentioned in the title) and the timestamp of when the movie was rated.
The timestamp first has to be transformed into a year of rating and the
year of release has to be extracted from the title:

\begin{Shaded}
\begin{Highlighting}[]
\NormalTok{edx_year <-}\StringTok{ }\NormalTok{edx }\OperatorTok\StringTok{ }\KeywordTok{mutate}\NormalTok{(}\DataTypeTok{rate_year =} \KeywordTok{year}\NormalTok{(}\KeywordTok{as_datetime}\NormalTok{(timestamp)))}
\NormalTok{edx_year <-}\StringTok{ }\NormalTok{edx_year }\OperatorTok\StringTok{ }\KeywordTok{mutate}\NormalTok{(}\DataTypeTok{title =} \KeywordTok{str_replace}\NormalTok{(title,}\StringTok{"^(.+)}\CharTok{\textbackslash{}\textbackslash{}}\StringTok{s}\CharTok{\textbackslash{}\textbackslash{}}\StringTok{((}\CharTok{\textbackslash{}\textbackslash{}}\StringTok{d\{4\})}\CharTok{\textbackslash{}\textbackslash{}}\StringTok{)$"}\NormalTok{,}\StringTok{"}\CharTok{\textbackslash{}\textbackslash{}}\StringTok{1__}\CharTok{\textbackslash{}\textbackslash{}}\StringTok{2"}\NormalTok{ )) }\OperatorTok\StringTok{ }
\StringTok{  }\KeywordTok{separate}\NormalTok{(title,}\KeywordTok{c}\NormalTok{(}\StringTok{"title"}\NormalTok{,}\StringTok{"release_year"}\NormalTok{),}\StringTok{"__"}\NormalTok{) }\OperatorTok
\StringTok{  }\KeywordTok{select}\NormalTok{(}\OperatorTok{-}\NormalTok{timestamp) }
\NormalTok{edx_year <-}\StringTok{ }\NormalTok{edx_year }\OperatorTok\StringTok{ }\KeywordTok{mutate}\NormalTok{(}\DataTypeTok{movieage_at_rating=} \KeywordTok{as.numeric}\NormalTok{(rate_year)}\OperatorTok{-}\KeywordTok{as.numeric}\NormalTok{(release_year), }\DataTypeTok{release_year =} \KeywordTok{as.numeric}\NormalTok{(release_year))}
\KeywordTok{kable}\NormalTok{(}\KeywordTok{head}\NormalTok{(edx_year))}
\end{Highlighting}
\end{Shaded}

\begin{longtable}[]{@{}rrrlrlrr@{}}
\toprule
userId & movieId & rating & title & release\_year & genres & rate\_year
& movieage\_at\_rating\tabularnewline
\midrule
\endhead
1 & 122 & 5 & Boomerang & 1992 & Comedy\textbar Romance & 1996 &
4\tabularnewline
1 & 185 & 5 & Net, The & 1995 & Action\textbar Crime\textbar Thriller &
1996 & 1\tabularnewline
1 & 292 & 5 & Outbreak & 1995 &
Action\textbar Drama\textbar Sci-Fi\textbar Thriller & 1996 &
1\tabularnewline
1 & 316 & 5 & Stargate & 1994 & Action\textbar Adventure\textbar Sci-Fi
& 1996 & 2\tabularnewline
1 & 329 & 5 & Star Trek: Generations & 1994 &
Action\textbar Adventure\textbar Drama\textbar Sci-Fi & 1996 &
2\tabularnewline
1 & 355 & 5 & Flintstones, The & 1994 &
Children\textbar Comedy\textbar Fantasy & 1996 & 2\tabularnewline
\bottomrule
\end{longtable}

Now the relation of the rating to the year of release and year of rating
can be visualized

\includegraphics{Movielens_project_files/figure-latex/boxplot-release-year-1.pdf}

The above boxplot indicates that the release year cause more variability
in the rating than the movie genres do. Thus, the release year could be
an indicator for the rating.

\includegraphics{Movielens_project_files/figure-latex/boxplot-release-year-scatterplot-1.pdf}

The lineplot shows more clearly the relationship between the average
rating for every release year and the release year itself. Movies before
1980 have been rated higher in average than movies after 1980.

\includegraphics{Movielens_project_files/figure-latex/boxplot-movieage-1.pdf}

Also the age of a movie influences how users rate that movie. The
scatter plot underneath shows the relationship.

\includegraphics{Movielens_project_files/figure-latex/movie-age-scatterplot-1.pdf}

The plot shows that older movies generally get higher ratings than
younger movies.

\hypertarget{summary}{%
\subsubsection{Summary}\label{summary}}

The above data exploration of the edx dataset fraction of the 10M
movielens dataset has shown that certain features in the dataset have an
influence on the rating. Thus, an algorithm which takes the
relationships between the ratings and the other features into account
should predict unknown ratings better in comparison to using the overall
rating average as the only predictor.

\includegraphics{Movielens_project_files/figure-latex/summary-data-exploration-1.pdf}

The above plot summarizes the 10M Movielens data exploration. It shows
that the rating variability is influenced much stronger by the UserId
and the MovieId than by genre, movieage or release year. Therefore, the
UserId and MovieId should certainly be taken into account when
developing a recommendation algorithm.

An important side note has to be made to the above data analysis. Namely
that only relationships between the ratings and the features explicitly
mentioned in the dataset have been investigated. It is, however, likely
that other unknowns features also have a strong influence on the
ratings.

\hypertarget{recommendation-algorithm-development}{%
\subsection{Recommendation algorithm
development}\label{recommendation-algorithm-development}}

\hypertarget{strategy}{%
\subsubsection{Strategy}\label{strategy}}

The data exploration has shown that UserId and MovieId have a strong
influence on the rating. In contrast, movieage at rating, release year
and the genre have a smaller effect on the rating variability. Thus, in
the first place a baseline algorithm model will be constructed, which
takes the movie and user effect into account. In this baseline model the
movie and user effect will be regularized to ensure that users and
movies with few ratings do not have a too strong contribution in the
model. The first few steps in the development of the baseline model are
based on the instructions in the ``Recommendation Systems'' chapter of
the text book
(\url{https://rafalab.github.io/dsbook/large-datasets.html\#recommendation-systems}).

The residuals remaining after the baseline model will be the basis for
matrix factorization. This method has been used by the winners of the
Netflix challenge to identify hidden patterns between movies and users.
In general, matrix factorization tries to separate the rating matrix
into a user embedded matrix and a movie embedded matrix. Using a
training set, the embeddings of the user embedded and movie embedded
matrices are learned to best represent the rating matrix again. In this
way, effects such as movieage at rating, release year and genre
information, as well as many other unknown features are included in the
recommender model if not already implicitly taken into account in the
baseline model. A more thorough description of matrix factorization can
be found here
(\url{https://developers.google.com/machine-learning/recommendation/collaborative/matrix}).
There are a few r packages which can perform matrix factorization, but
here the ``recosystem'' package
(\url{https://cran.r-project.org/web/packages/recosystem/vignettes/introduction.html})
was chosen due to its ease of use, reduced memory usage and the
relatively low computational power it requires. The r recosystem package
is a wrapper for the high-performance C++ LIMBF library which uses the
parallel stochastic gradient descent algorithm for optimizing the user
embedded matrix and the movie embedded matrix.

The edx dataset has been split up in a training set (edx\_train\_set)
and a test set (edx\_test\_set). The training set will be used to train
the models during the recommendation algorithm development, while the
test set will test these developed models. The testing will be performed
by taking the root-mean-square error (RMSE) between the predicted
ratings and the true ratings for the test set. The lower the RMSE, the
better the recommender algorithm.

\begin{Shaded}
\begin{Highlighting}[]
\CommentTok{# RMSE function}
\NormalTok{RMSE <-}\StringTok{ }\ControlFlowTok{function}\NormalTok{(ratings_true, ratings_predicted)\{}
  \KeywordTok{sqrt}\NormalTok{(}\KeywordTok{mean}\NormalTok{((ratings_true }\OperatorTok{-}\StringTok{ }\NormalTok{ratings_predicted)}\OperatorTok{^}\DecValTok{2}\NormalTok{))}
\NormalTok{\}}
\end{Highlighting}
\end{Shaded}

\hypertarget{model-1-average-ratings-for-each-movie}{%
\subsubsection{Model 1: average ratings for each
movie}\label{model-1-average-ratings-for-each-movie}}

As a first model just the ratings themselves are taken into account. The
RMSE can then be minimized by using the overall average of all ratings
(3.51248), which yield an RMSE of 1.0606.

\begin{Shaded}
\begin{Highlighting}[]
\CommentTok{# Model 1: average rating }
\CommentTok{# average rating of all movies for all users to predict the rating for user u and movie i}
\NormalTok{model_}\DecValTok{1}\NormalTok{_prediction <-}\StringTok{ }\KeywordTok{mean}\NormalTok{(edx_train_set}\OperatorTok{$}\NormalTok{rating)}
\NormalTok{model_}\DecValTok{1}\NormalTok{_prediction}
\end{Highlighting}
\end{Shaded}

\begin{verbatim}
## [1] 3.51248
\end{verbatim}

\begin{Shaded}
\begin{Highlighting}[]
\NormalTok{model_}\DecValTok{1}\NormalTok{_RMSE <-}\StringTok{ }\KeywordTok{RMSE}\NormalTok{(edx_test_set}\OperatorTok{$}\NormalTok{rating, model_}\DecValTok{1}\NormalTok{_prediction)}
\NormalTok{model_}\DecValTok{1}\NormalTok{_RMSE}
\end{Highlighting}
\end{Shaded}

\begin{verbatim}
## [1] 1.060635
\end{verbatim}

\begin{Shaded}
\begin{Highlighting}[]
\NormalTok{all_RMSEs <-}\StringTok{ }\KeywordTok{data_frame}\NormalTok{(}\DataTypeTok{method =} \StringTok{"Model 1 : overall average"}\NormalTok{, }\DataTypeTok{RMSE =}\NormalTok{ model_}\DecValTok{1}\NormalTok{_RMSE)}
\NormalTok{all_RMSEs }\OperatorTok\StringTok{ }\NormalTok{knitr}\OperatorTok{::}\KeywordTok{kable}\NormalTok{()}
\end{Highlighting}
\end{Shaded}

\begin{longtable}[]{@{}lr@{}}
\toprule
method & RMSE\tabularnewline
\midrule
\endhead
Model 1 : overall average & 1.060635\tabularnewline
\bottomrule
\end{longtable}

\hypertarget{model-2-modeling-the-movie-effect-by-adding-a-movie-bias-m_bias}{%
\subsubsection{Model 2: modeling the movie effect by adding a movie bias
(m\_bias)}\label{model-2-modeling-the-movie-effect-by-adding-a-movie-bias-m_bias}}

The characteristics of a movie have a certain influence on the ratings
of that movie. In the recommendation model the movie effect can be taken
into account by adding a bias for each movie (m\_bias). This bias
corresponds to the difference between the average rating of that
specific movie to the overall average rating of all movies.

\begin{Shaded}
\begin{Highlighting}[]
\CommentTok{# Model 2: average rating + movie bias}
\CommentTok{# inclusion of a movie bias (m_bias) since movies can in average be rated higher or lower than the overall average}
\NormalTok{avg_rating <-}\StringTok{ }\KeywordTok{mean}\NormalTok{(edx_train_set}\OperatorTok{$}\NormalTok{rating)}

\NormalTok{movie_avgs <-}\StringTok{ }\NormalTok{edx_train_set }\OperatorTok\StringTok{ }
\StringTok{  }\KeywordTok{group_by}\NormalTok{(movieId) }\OperatorTok\StringTok{ }
\StringTok{  }\KeywordTok{summarize}\NormalTok{(}\DataTypeTok{m_bias =} \KeywordTok{mean}\NormalTok{(rating }\OperatorTok{-}\StringTok{ }\NormalTok{avg_rating))}

\NormalTok{model_}\DecValTok{2}\NormalTok{_prediction <-}\StringTok{ }\NormalTok{avg_rating }\OperatorTok{+}\StringTok{ }\NormalTok{edx_test_set }\OperatorTok
\StringTok{  }\KeywordTok{left_join}\NormalTok{(movie_avgs, }\DataTypeTok{by=}\StringTok{'movieId'}\NormalTok{) }\OperatorTok
\StringTok{  }\NormalTok{.}\OperatorTok{$}\NormalTok{m_bias}

\NormalTok{model_}\DecValTok{2}\NormalTok{_RMSE <-}\StringTok{ }\KeywordTok{RMSE}\NormalTok{(model_}\DecValTok{2}\NormalTok{_prediction, edx_test_set}\OperatorTok{$}\NormalTok{rating)}
\NormalTok{all_RMSEs <-}\StringTok{ }\KeywordTok{bind_rows}\NormalTok{(all_RMSEs,}
                          \KeywordTok{data_frame}\NormalTok{(}\DataTypeTok{method=}\StringTok{"Model 2: movie bias"}\NormalTok{,}
                                     \DataTypeTok{RMSE =}\NormalTok{ model_}\DecValTok{2}\NormalTok{_RMSE))}
\NormalTok{all_RMSEs }\OperatorTok\StringTok{ }\NormalTok{knitr}\OperatorTok{::}\KeywordTok{kable}\NormalTok{()}
\end{Highlighting}
\end{Shaded}

\begin{longtable}[]{@{}lr@{}}
\toprule
method & RMSE\tabularnewline
\midrule
\endhead
Model 1 : overall average & 1.0606351\tabularnewline
Model 2: movie bias & 0.9440709\tabularnewline
\bottomrule
\end{longtable}

\hypertarget{model-3-modeling-the-user-effect-by-adding-a-user-bias-in-addition-to-the-movie-bias-m_bias-u_bias}{%
\subsubsection{Model 3: modeling the user effect by adding a user bias
in addition to the movie bias (m\_bias +
u\_bias)}\label{model-3-modeling-the-user-effect-by-adding-a-user-bias-in-addition-to-the-movie-bias-m_bias-u_bias}}

The characteristics of a user can also influence the rating of a
particular movie. There are for example optimistic users who like to
give higher scores in comparison to other users. The user effect is
accommodated by including a user bias (u\_bias) to model 2.

\begin{Shaded}
\begin{Highlighting}[]
\CommentTok{# Model 3: average rating + movie bias + user bias}
\CommentTok{# inclusion of a user bias (u_bias) since users can in average rate movies higher or lower than the overall average}
\NormalTok{avg_rating <-}\StringTok{ }\KeywordTok{mean}\NormalTok{(edx_train_set}\OperatorTok{$}\NormalTok{rating)}

\NormalTok{user_avgs <-}\StringTok{ }\NormalTok{edx_train_set }\OperatorTok\StringTok{ }
\StringTok{  }\KeywordTok{left_join}\NormalTok{(movie_avgs, }\DataTypeTok{by=}\StringTok{'movieId'}\NormalTok{) }\OperatorTok
\StringTok{  }\KeywordTok{group_by}\NormalTok{(userId) }\OperatorTok\StringTok{ }
\StringTok{  }\KeywordTok{summarize}\NormalTok{(}\DataTypeTok{u_bias =} \KeywordTok{mean}\NormalTok{(rating }\OperatorTok{-}\StringTok{ }\NormalTok{avg_rating }\OperatorTok{-}\StringTok{ }\NormalTok{m_bias))}

\NormalTok{model_}\DecValTok{3}\NormalTok{_prediction <-}\StringTok{ }\NormalTok{edx_test_set }\OperatorTok
\StringTok{  }\KeywordTok{left_join}\NormalTok{(movie_avgs, }\DataTypeTok{by=}\StringTok{'movieId'}\NormalTok{) }\OperatorTok
\StringTok{  }\KeywordTok{left_join}\NormalTok{(user_avgs, }\DataTypeTok{by=}\StringTok{'userId'}\NormalTok{) }\OperatorTok
\StringTok{  }\KeywordTok{mutate}\NormalTok{(}\DataTypeTok{prediction =}\NormalTok{ avg_rating }\OperatorTok{+}\StringTok{ }\NormalTok{m_bias }\OperatorTok{+}\StringTok{ }\NormalTok{u_bias) }\OperatorTok
\StringTok{  }\NormalTok{.}\OperatorTok{$}\NormalTok{prediction}
  

\NormalTok{model_}\DecValTok{3}\NormalTok{_RMSE <-}\StringTok{ }\KeywordTok{RMSE}\NormalTok{(model_}\DecValTok{3}\NormalTok{_prediction, edx_test_set}\OperatorTok{$}\NormalTok{rating)}

\NormalTok{all_RMSEs <-}\StringTok{ }\KeywordTok{bind_rows}\NormalTok{(all_RMSEs,}
                       \KeywordTok{data_frame}\NormalTok{(}\DataTypeTok{method=}\StringTok{"Model 3: movie bias + user bias"}\NormalTok{,}
                                  \DataTypeTok{RMSE =}\NormalTok{ model_}\DecValTok{3}\NormalTok{_RMSE))}
\NormalTok{all_RMSEs }\OperatorTok\StringTok{ }\NormalTok{knitr}\OperatorTok{::}\KeywordTok{kable}\NormalTok{()}
\end{Highlighting}
\end{Shaded}

\begin{longtable}[]{@{}lr@{}}
\toprule
method & RMSE\tabularnewline
\midrule
\endhead
Model 1 : overall average & 1.0606351\tabularnewline
Model 2: movie bias & 0.9440709\tabularnewline
Model 3: movie bias + user bias & 0.8664802\tabularnewline
\bottomrule
\end{longtable}

\hypertarget{model-4-modeling-the-movie-effect-by-regularizing-the-movie-bias-m_bias-using-a-parameter-lambda-m_lambda}{%
\subsubsection{Model 4: modeling the movie effect by regularizing the
movie bias (m\_bias) using a parameter lambda
(m\_lambda)}\label{model-4-modeling-the-movie-effect-by-regularizing-the-movie-bias-m_bias-using-a-parameter-lambda-m_lambda}}

Preliminary data exploration has shown tha many obscure movies were only
rated one or a few times. This led to obscure movies being very highly
or very lowly rated in average. However, one or a few ratings are
generally not sufficient to confidently predict future ratings.

If such small sample sizes are not taken into account, the
recommendation algorithm can become overtrained. Small sample sizes can
be accounted for by adding a penalty to the movie bias estimates, which
increases with smaller sample sizes. This method is called
regularization.

How the penalty should vary with the sample size can be modeled using
the parameter lambda (m\_lambda). In model 4 depicted below a sequence
of lambdas is tested to verify which lambda results in the lowest RMSE.

\begin{Shaded}
\begin{Highlighting}[]
\CommentTok{# Model 4: average rating + regularized movie bias }
\CommentTok{# inclusion of a regularized movie bias (reg_m_bias) to account for movies which were not rated often}

\CommentTok{# Optimize the lambda parameter in the regularization formula}
\NormalTok{lambdas <-}\StringTok{ }\KeywordTok{seq}\NormalTok{(}\DecValTok{0}\NormalTok{, }\DecValTok{10}\NormalTok{, }\FloatTok{0.25}\NormalTok{)}
\NormalTok{avg_rating <-}\StringTok{ }\KeywordTok{mean}\NormalTok{(edx_train_set}\OperatorTok{$}\NormalTok{rating)}
\NormalTok{sum_rating_minus_avg <-}\StringTok{ }\NormalTok{edx_train_set }\OperatorTok\StringTok{ }
\StringTok{  }\KeywordTok{group_by}\NormalTok{(movieId) }\OperatorTok\StringTok{ }
\StringTok{  }\KeywordTok{summarize}\NormalTok{(}\DataTypeTok{s =} \KeywordTok{sum}\NormalTok{(rating }\OperatorTok{-}\StringTok{ }\NormalTok{avg_rating), }\DataTypeTok{n_i =} \KeywordTok{n}\NormalTok{())}

\NormalTok{RMSEs <-}\StringTok{ }\KeywordTok{sapply}\NormalTok{(lambdas, }\ControlFlowTok{function}\NormalTok{(l)\{}
\NormalTok{  predicted_ratings <-}\StringTok{ }\NormalTok{edx_test_set }\OperatorTok\StringTok{ }
\StringTok{    }\KeywordTok{left_join}\NormalTok{(sum_rating_minus_avg, }\DataTypeTok{by=}\StringTok{'movieId'}\NormalTok{) }\OperatorTok\StringTok{ }
\StringTok{    }\KeywordTok{mutate}\NormalTok{(}\DataTypeTok{m_bias =}\NormalTok{ s}\OperatorTok{/}\NormalTok{(n_i}\OperatorTok{+}\NormalTok{l)) }\OperatorTok
\StringTok{    }\KeywordTok{mutate}\NormalTok{(}\DataTypeTok{prediction =}\NormalTok{ avg_rating }\OperatorTok{+}\StringTok{ }\NormalTok{m_bias) }\OperatorTok
\StringTok{    }\NormalTok{.}\OperatorTok{$}\NormalTok{pred}
  \KeywordTok{return}\NormalTok{(}\KeywordTok{RMSE}\NormalTok{(predicted_ratings, edx_test_set}\OperatorTok{$}\NormalTok{rating))}
\NormalTok{\})}
\KeywordTok{qplot}\NormalTok{(lambdas, RMSEs)}
\end{Highlighting}
\end{Shaded}

\includegraphics{Movielens_project_files/figure-latex/unnamed-chunk-5-1.pdf}

\begin{Shaded}
\begin{Highlighting}[]
\NormalTok{m_lambda <-}\StringTok{ }\NormalTok{lambdas[}\KeywordTok{which.min}\NormalTok{(RMSEs)]}

\NormalTok{m_lambda}
\end{Highlighting}
\end{Shaded}

\begin{verbatim}
## [1] 1.75
\end{verbatim}

\begin{Shaded}
\begin{Highlighting}[]
\NormalTok{all_RMSEs <-}\StringTok{ }\KeywordTok{bind_rows}\NormalTok{(all_RMSEs,}
                       \KeywordTok{data_frame}\NormalTok{(}\DataTypeTok{method=}\StringTok{"Model 4: regularized movie bias"}\NormalTok{,}
                                  \DataTypeTok{RMSE =} \KeywordTok{min}\NormalTok{(RMSEs)))}
\NormalTok{all_RMSEs }\OperatorTok\StringTok{ }\NormalTok{knitr}\OperatorTok{::}\KeywordTok{kable}\NormalTok{()}
\end{Highlighting}
\end{Shaded}

\begin{longtable}[]{@{}lr@{}}
\toprule
method & RMSE\tabularnewline
\midrule
\endhead
Model 1 : overall average & 1.0606351\tabularnewline
Model 2: movie bias & 0.9440709\tabularnewline
Model 3: movie bias + user bias & 0.8664802\tabularnewline
Model 4: regularized movie bias & 0.9440293\tabularnewline
\bottomrule
\end{longtable}

\hypertarget{model-5-modeling-the-user-effect-by-regularizing-the-user-bias-u_bias-using-a-parameter-lambda-u_lambda}{%
\subsubsection{Model 5: modeling the user effect by regularizing the
user bias (u\_bias) using a parameter lambda
(u\_lambda)}\label{model-5-modeling-the-user-effect-by-regularizing-the-user-bias-u_bias-using-a-parameter-lambda-u_lambda}}

Regularization can be applied as well to the user bias in a similar way
to the movie bias. Indeed, a user who has rated a few movies very
highly, will not necessarily rate every movie very highly. He may still
rate movies in average highly but this cannot be derived with high
certainty from the small sample of movies the user has rated.

The lambda parameter was here also used to optimize the penalty to the
user bias. The optimized lambda for the user effect is denoted as
u\_lambda.

\begin{Shaded}
\begin{Highlighting}[]
\CommentTok{# Model 5: regularized user bias}
\CommentTok{#Include a regularized user bias to lower the importance of the user bias for users who did not yet rate a lot of movies}

\CommentTok{#Optimize the lambda parameter in the regularization formula}
\NormalTok{lambdas <-}\StringTok{ }\KeywordTok{seq}\NormalTok{(}\DecValTok{0}\NormalTok{, }\DecValTok{10}\NormalTok{, }\FloatTok{0.25}\NormalTok{)}
\NormalTok{avg_rating <-}\StringTok{ }\KeywordTok{mean}\NormalTok{(edx_train_set}\OperatorTok{$}\NormalTok{rating)}
\NormalTok{sum_rating_minus_avg <-}\StringTok{ }\NormalTok{edx_train_set }\OperatorTok\StringTok{ }
\StringTok{  }\KeywordTok{group_by}\NormalTok{(userId) }\OperatorTok\StringTok{ }
\StringTok{  }\KeywordTok{summarize}\NormalTok{(}\DataTypeTok{s =} \KeywordTok{sum}\NormalTok{(rating }\OperatorTok{-}\StringTok{ }\NormalTok{avg_rating), }\DataTypeTok{n_i =} \KeywordTok{n}\NormalTok{())}

\NormalTok{RMSEs <-}\StringTok{ }\KeywordTok{sapply}\NormalTok{(lambdas, }\ControlFlowTok{function}\NormalTok{(l)\{}
\NormalTok{  predicted_ratings <-}\StringTok{ }\NormalTok{edx_test_set }\OperatorTok\StringTok{ }
\StringTok{    }\KeywordTok{left_join}\NormalTok{(sum_rating_minus_avg, }\DataTypeTok{by=}\StringTok{'userId'}\NormalTok{) }\OperatorTok\StringTok{ }
\StringTok{    }\KeywordTok{mutate}\NormalTok{(}\DataTypeTok{u_bias =}\NormalTok{ s}\OperatorTok{/}\NormalTok{(n_i}\OperatorTok{+}\NormalTok{l)) }\OperatorTok
\StringTok{    }\KeywordTok{mutate}\NormalTok{(}\DataTypeTok{prediction =}\NormalTok{ avg_rating }\OperatorTok{+}\StringTok{ }\NormalTok{u_bias) }\OperatorTok
\StringTok{    }\NormalTok{.}\OperatorTok{$}\NormalTok{pred}
  \KeywordTok{return}\NormalTok{(}\KeywordTok{RMSE}\NormalTok{(predicted_ratings, edx_test_set}\OperatorTok{$}\NormalTok{rating))}
\NormalTok{\})}
\KeywordTok{qplot}\NormalTok{(lambdas, RMSEs)}
\end{Highlighting}
\end{Shaded}

\includegraphics{Movielens_project_files/figure-latex/unnamed-chunk-6-1.pdf}

\begin{Shaded}
\begin{Highlighting}[]
\NormalTok{u_lambda <-}\StringTok{ }\NormalTok{lambdas[}\KeywordTok{which.min}\NormalTok{(RMSEs)]}

\NormalTok{u_lambda}
\end{Highlighting}
\end{Shaded}

\begin{verbatim}
## [1] 5.5
\end{verbatim}

\begin{Shaded}
\begin{Highlighting}[]
\NormalTok{all_RMSEs <-}\StringTok{ }\KeywordTok{bind_rows}\NormalTok{(all_RMSEs,}
                       \KeywordTok{data_frame}\NormalTok{(}\DataTypeTok{method=}\StringTok{"Model 5: regularized user bias"}\NormalTok{,}
                                  \DataTypeTok{RMSE =} \KeywordTok{min}\NormalTok{(RMSEs)))}
\NormalTok{all_RMSEs }\OperatorTok\StringTok{ }\NormalTok{knitr}\OperatorTok{::}\KeywordTok{kable}\NormalTok{()}
\end{Highlighting}
\end{Shaded}

\begin{longtable}[]{@{}lr@{}}
\toprule
method & RMSE\tabularnewline
\midrule
\endhead
Model 1 : overall average & 1.0606351\tabularnewline
Model 2: movie bias & 0.9440709\tabularnewline
Model 3: movie bias + user bias & 0.8664802\tabularnewline
Model 4: regularized movie bias & 0.9440293\tabularnewline
Model 5: regularized user bias & 0.9785199\tabularnewline
\bottomrule
\end{longtable}

\hypertarget{model-6-modeling-movie-effect-and-user-effect-using-the-regularized-biases-and-lambdas-from-the-previous-models}{%
\subsubsection{Model 6: modeling movie effect and user effect using the
regularized biases and lambdas from the previous
models}\label{model-6-modeling-movie-effect-and-user-effect-using-the-regularized-biases-and-lambdas-from-the-previous-models}}

In model 6, model 4 and 5 are combined so that a regularized bias for
the movie and user effect is used. The lambdas optimized in each model
are being used: 5.5 for the movie bias and 1.75 for the user bias.

\begin{Shaded}
\begin{Highlighting}[]
\CommentTok{# Model 6: regularized movie bias + user bias}
\CommentTok{#Now include both a regularized user bias with lambda 5.5 and a regularized movie bias with lambda 1.75}
\NormalTok{u_lambda <-}\StringTok{ }\FloatTok{5.5}
\NormalTok{m_lambda <-}\StringTok{ }\FloatTok{1.75}

\NormalTok{avg_rating <-}\StringTok{ }\KeywordTok{mean}\NormalTok{(edx_train_set}\OperatorTok{$}\NormalTok{rating)}

\CommentTok{# regularized movie bias based on the lambda calculated in model 4}
\NormalTok{reg_movie_bias <-}\StringTok{ }\NormalTok{edx_train_set }\OperatorTok\StringTok{ }
\StringTok{  }\KeywordTok{group_by}\NormalTok{(movieId) }\OperatorTok\StringTok{ }
\StringTok{  }\KeywordTok{summarize}\NormalTok{(}\DataTypeTok{reg_m_bias =}\NormalTok{ (}\KeywordTok{sum}\NormalTok{(rating }\OperatorTok{-}\StringTok{ }\NormalTok{avg_rating))}\OperatorTok{/}\NormalTok{(}\KeywordTok{n}\NormalTok{()}\OperatorTok{+}\NormalTok{m_lambda))}

\CommentTok{# regularized user bias based on the lambda calculated in model 5}
\NormalTok{reg_user_bias <-}\StringTok{ }\NormalTok{edx_train_set }\OperatorTok\StringTok{ }
\StringTok{  }\KeywordTok{left_join}\NormalTok{(reg_movie_bias, }\DataTypeTok{by=}\StringTok{'movieId'}\NormalTok{) }\OperatorTok
\StringTok{  }\KeywordTok{group_by}\NormalTok{(userId) }\OperatorTok\StringTok{ }
\StringTok{  }\KeywordTok{summarize}\NormalTok{(}\DataTypeTok{reg_u_bias =}\NormalTok{ (}\KeywordTok{sum}\NormalTok{(rating }\OperatorTok{-}\StringTok{ }\NormalTok{avg_rating }\OperatorTok{-}\StringTok{ }\NormalTok{reg_m_bias))}\OperatorTok{/}\NormalTok{(}\KeywordTok{n}\NormalTok{()}\OperatorTok{+}\NormalTok{u_lambda))}
 
\NormalTok{model_}\DecValTok{6}\NormalTok{_prediction <-}\StringTok{ }\NormalTok{edx_test_set }\OperatorTok
\StringTok{  }\KeywordTok{left_join}\NormalTok{(reg_movie_bias, }\DataTypeTok{by=}\StringTok{'movieId'}\NormalTok{) }\OperatorTok
\StringTok{  }\KeywordTok{left_join}\NormalTok{(reg_user_bias, }\DataTypeTok{by=}\StringTok{'userId'}\NormalTok{) }\OperatorTok
\StringTok{  }\KeywordTok{mutate}\NormalTok{(}\DataTypeTok{prediction =}\NormalTok{ avg_rating }\OperatorTok{+}\StringTok{ }\NormalTok{reg_m_bias }\OperatorTok{+}\StringTok{ }\NormalTok{reg_u_bias) }\OperatorTok
\StringTok{  }\NormalTok{.}\OperatorTok{$}\NormalTok{prediction}

\NormalTok{model_}\DecValTok{6}\NormalTok{_RMSE <-}\StringTok{ }\KeywordTok{RMSE}\NormalTok{(model_}\DecValTok{6}\NormalTok{_prediction, edx_test_set}\OperatorTok{$}\NormalTok{rating)}

\NormalTok{all_RMSEs <-}\StringTok{ }\KeywordTok{bind_rows}\NormalTok{(all_RMSEs,}
                       \KeywordTok{data_frame}\NormalTok{(}\DataTypeTok{method=}\StringTok{"Model 6: regularized movie bias + regularized user bias"}\NormalTok{,}
                                  \DataTypeTok{RMSE =}\NormalTok{ model_}\DecValTok{6}\NormalTok{_RMSE))}
\NormalTok{all_RMSEs }\OperatorTok\StringTok{ }\NormalTok{knitr}\OperatorTok{::}\KeywordTok{kable}\NormalTok{()}
\end{Highlighting}
\end{Shaded}

\begin{longtable}[]{@{}lr@{}}
\toprule
method & RMSE\tabularnewline
\midrule
\endhead
Model 1 : overall average & 1.0606351\tabularnewline
Model 2: movie bias & 0.9440709\tabularnewline
Model 3: movie bias + user bias & 0.8664802\tabularnewline
Model 4: regularized movie bias & 0.9440293\tabularnewline
Model 5: regularized user bias & 0.9785199\tabularnewline
Model 6: regularized movie bias + regularized user bias &
0.8658766\tabularnewline
\bottomrule
\end{longtable}

\hypertarget{model-7-modeling-movie-effect-and-user-effect-using-regularized-biases-by-optimizing-lambda-simultaneously-for-both-effects.}{%
\subsubsection{Model 7: modeling movie effect and user effect using
regularized biases by optimizing lambda simultaneously for both
effects.}\label{model-7-modeling-movie-effect-and-user-effect-using-regularized-biases-by-optimizing-lambda-simultaneously-for-both-effects.}}

In model 7 the same lambda will be used for regularizing both the movie
and user effect.

\begin{Shaded}
\begin{Highlighting}[]
\CommentTok{# Model 7: regularized movie bias + user bias 2}
\CommentTok{# In the second type the lambda is simultaneously optimized for the movie bias and the user bias}

\CommentTok{# First optimize the lambda}
\NormalTok{avg_rating <-}\StringTok{ }\KeywordTok{mean}\NormalTok{(edx_train_set}\OperatorTok{$}\NormalTok{rating)}
\NormalTok{lambdas <-}\StringTok{ }\KeywordTok{seq}\NormalTok{(}\DecValTok{0}\NormalTok{, }\DecValTok{10}\NormalTok{, }\FloatTok{0.25}\NormalTok{)}

\NormalTok{RMSEs <-}\StringTok{ }\KeywordTok{sapply}\NormalTok{(lambdas, }\ControlFlowTok{function}\NormalTok{(um_lambda)\{}
\NormalTok{  reg_movie_bias <-}\StringTok{ }\NormalTok{edx_train_set }\OperatorTok\StringTok{ }
\StringTok{    }\KeywordTok{group_by}\NormalTok{(movieId) }\OperatorTok\StringTok{ }
\StringTok{    }\KeywordTok{summarize}\NormalTok{(}\DataTypeTok{reg_m_bias =}\NormalTok{ (}\KeywordTok{sum}\NormalTok{(rating }\OperatorTok{-}\StringTok{ }\NormalTok{avg_rating))}\OperatorTok{/}\NormalTok{(}\KeywordTok{n}\NormalTok{()}\OperatorTok{+}\NormalTok{um_lambda))}
\NormalTok{  reg_user_bias <-}\StringTok{ }\NormalTok{edx_train_set }\OperatorTok\StringTok{ }
\StringTok{    }\KeywordTok{left_join}\NormalTok{(reg_movie_bias, }\DataTypeTok{by=}\StringTok{'movieId'}\NormalTok{) }\OperatorTok
\StringTok{    }\KeywordTok{group_by}\NormalTok{(userId) }\OperatorTok\StringTok{ }
\StringTok{    }\KeywordTok{summarize}\NormalTok{(}\DataTypeTok{reg_u_bias =}\NormalTok{ (}\KeywordTok{sum}\NormalTok{(rating }\OperatorTok{-}\StringTok{ }\NormalTok{avg_rating }\OperatorTok{-}\StringTok{ }\NormalTok{reg_m_bias))}\OperatorTok{/}\NormalTok{(}\KeywordTok{n}\NormalTok{()}\OperatorTok{+}\NormalTok{um_lambda))}
\NormalTok{  predicted_ratings <-}\StringTok{ }\NormalTok{edx_test_set }\OperatorTok
\StringTok{    }\KeywordTok{left_join}\NormalTok{(reg_movie_bias, }\DataTypeTok{by=}\StringTok{'movieId'}\NormalTok{) }\OperatorTok
\StringTok{    }\KeywordTok{left_join}\NormalTok{(reg_user_bias, }\DataTypeTok{by=}\StringTok{'userId'}\NormalTok{) }\OperatorTok
\StringTok{    }\KeywordTok{mutate}\NormalTok{(}\DataTypeTok{prediction =}\NormalTok{ avg_rating }\OperatorTok{+}\StringTok{ }\NormalTok{reg_m_bias }\OperatorTok{+}\StringTok{ }\NormalTok{reg_u_bias) }\OperatorTok
\StringTok{    }\NormalTok{.}\OperatorTok{$}\NormalTok{prediction}
  \KeywordTok{return}\NormalTok{(}\KeywordTok{RMSE}\NormalTok{(predicted_ratings, edx_test_set}\OperatorTok{$}\NormalTok{rating))}
\NormalTok{\})}

\KeywordTok{qplot}\NormalTok{(lambdas, RMSEs)}
\end{Highlighting}
\end{Shaded}

\includegraphics{Movielens_project_files/figure-latex/unnamed-chunk-8-1.pdf}

\begin{Shaded}
\begin{Highlighting}[]
\NormalTok{um_lambda <-}\StringTok{ }\NormalTok{lambdas[}\KeywordTok{which.min}\NormalTok{(RMSEs)]}
\NormalTok{um_lambda}
\end{Highlighting}
\end{Shaded}

\begin{verbatim}
## [1] 4.75
\end{verbatim}

\begin{Shaded}
\begin{Highlighting}[]
\NormalTok{model_}\DecValTok{7}\NormalTok{_RMSE <-}\StringTok{ }\KeywordTok{min}\NormalTok{(RMSEs)}
\NormalTok{all_RMSEs <-}\StringTok{ }\KeywordTok{bind_rows}\NormalTok{(all_RMSEs,}
                       \KeywordTok{data_frame}\NormalTok{(}\DataTypeTok{method=}\StringTok{"Model 7: regularized movie bias + regularized user bias 2"}\NormalTok{,}
                                  \DataTypeTok{RMSE =}\NormalTok{ model_}\DecValTok{7}\NormalTok{_RMSE))}
\NormalTok{all_RMSEs }\OperatorTok\StringTok{ }\NormalTok{knitr}\OperatorTok{::}\KeywordTok{kable}\NormalTok{()}
\end{Highlighting}
\end{Shaded}

\begin{longtable}[]{@{}lr@{}}
\toprule
method & RMSE\tabularnewline
\midrule
\endhead
Model 1 : overall average & 1.0606351\tabularnewline
Model 2: movie bias & 0.9440709\tabularnewline
Model 3: movie bias + user bias & 0.8664802\tabularnewline
Model 4: regularized movie bias & 0.9440293\tabularnewline
Model 5: regularized user bias & 0.9785199\tabularnewline
Model 6: regularized movie bias + regularized user bias &
0.8658766\tabularnewline
Model 7: regularized movie bias + regularized user bias 2 &
0.8658479\tabularnewline
\bottomrule
\end{longtable}

\hypertarget{model-8-matrix-factorization-on-residuals-of-model-7-baseline-model}{%
\subsubsection{Model 8: Matrix factorization on residuals of model 7
(baseline
model)}\label{model-8-matrix-factorization-on-residuals-of-model-7-baseline-model}}

Since model 7 exhibited the lowest RMSE the residuals from that model
were used for matrix factorization. The residuals are obtained after
subtracting the average rating, movie bias and user bias from each
rating.

\begin{Shaded}
\begin{Highlighting}[]
\CommentTok{# Model 8: Matrix factorization}
\CommentTok{# To further enhance the recommendation the residuals have to be modeled using matrix factorization,}
\CommentTok{# The matrix factorization will be performed using the recosystem package }
\CommentTok{# Model 7 will be used as baseline system to calculate the residuals }

\CommentTok{# Obtain the residuals using the lambda of model 7}
\NormalTok{um_lambda <-}\StringTok{ }\FloatTok{4.75}
\NormalTok{avg_rating <-}\StringTok{ }\KeywordTok{mean}\NormalTok{(edx_train_set}\OperatorTok{$}\NormalTok{rating)}

\NormalTok{reg_movie_bias <-}\StringTok{ }\NormalTok{edx_train_set }\OperatorTok\StringTok{ }
\StringTok{  }\KeywordTok{group_by}\NormalTok{(movieId) }\OperatorTok\StringTok{ }
\StringTok{  }\KeywordTok{summarize}\NormalTok{(}\DataTypeTok{reg_m_bias =}\NormalTok{ (}\KeywordTok{sum}\NormalTok{(rating }\OperatorTok{-}\StringTok{ }\NormalTok{avg_rating))}\OperatorTok{/}\NormalTok{(}\KeywordTok{n}\NormalTok{()}\OperatorTok{+}\NormalTok{um_lambda))}
\NormalTok{reg_user_bias <-}\StringTok{ }\NormalTok{edx_train_set }\OperatorTok\StringTok{ }
\StringTok{  }\KeywordTok{left_join}\NormalTok{(reg_movie_bias, }\DataTypeTok{by=}\StringTok{'movieId'}\NormalTok{) }\OperatorTok
\StringTok{  }\KeywordTok{group_by}\NormalTok{(userId) }\OperatorTok\StringTok{ }
\StringTok{  }\KeywordTok{summarize}\NormalTok{(}\DataTypeTok{reg_u_bias =}\NormalTok{ (}\KeywordTok{sum}\NormalTok{(rating }\OperatorTok{-}\StringTok{ }\NormalTok{avg_rating }\OperatorTok{-}\StringTok{ }\NormalTok{reg_m_bias))}\OperatorTok{/}\NormalTok{(}\KeywordTok{n}\NormalTok{()}\OperatorTok{+}\NormalTok{um_lambda))}

\NormalTok{predicted_ratings_m7 <-}\StringTok{ }
\StringTok{  }\NormalTok{edx_test_set }\OperatorTok\StringTok{ }
\StringTok{  }\KeywordTok{left_join}\NormalTok{(reg_movie_bias, }\DataTypeTok{by =} \StringTok{"movieId"}\NormalTok{) }\OperatorTok
\StringTok{  }\KeywordTok{left_join}\NormalTok{(reg_user_bias, }\DataTypeTok{by =} \StringTok{"userId"}\NormalTok{) }\OperatorTok
\StringTok{  }\KeywordTok{mutate}\NormalTok{(}\DataTypeTok{pred =}\NormalTok{ avg_rating }\OperatorTok{+}\StringTok{ }\NormalTok{reg_m_bias }\OperatorTok{+}\StringTok{ }\NormalTok{reg_u_bias)}\OperatorTok
\StringTok{  }\KeywordTok{pull}\NormalTok{(pred) }

\NormalTok{residuals_train_set <-}\StringTok{ }\NormalTok{edx_train_set }\OperatorTok\StringTok{ }
\StringTok{  }\KeywordTok{left_join}\NormalTok{(reg_movie_bias, }\DataTypeTok{by =} \StringTok{"movieId"}\NormalTok{) }\OperatorTok
\StringTok{  }\KeywordTok{left_join}\NormalTok{(reg_user_bias, }\DataTypeTok{by =} \StringTok{"userId"}\NormalTok{) }\OperatorTok
\StringTok{  }\KeywordTok{mutate}\NormalTok{(}\DataTypeTok{residual =}\NormalTok{ rating }\OperatorTok{-}\StringTok{ }\NormalTok{avg_rating }\OperatorTok{-}\StringTok{ }\NormalTok{reg_m_bias }\OperatorTok{-}\StringTok{ }\NormalTok{reg_u_bias) }\OperatorTok
\StringTok{  }\KeywordTok{select}\NormalTok{(userId, movieId, residual)}
\KeywordTok{head}\NormalTok{(residuals_train_set)}
\end{Highlighting}
\end{Shaded}

\begin{verbatim}
##   userId movieId   residual
## 1      1     185  0.6167235
## 2      1     292  0.3273172
## 3      1     316  0.3990895
## 4      1     329  0.4071122
## 5      1     356 -0.2640140
## 6      1     362  0.3091414
\end{verbatim}

These residuals are the basis for matrix factorization. For the
recosystem package both edx training and test set need to be transformed
into three-column matrices with users, movies and residuals/ratings as
column identifiers. These matrices are then temporarily written onto the
hard disk, using less RAM memory. Then, the Reco() function in the
recosystem package will be used to built a recommender object of which a
set of parameters will be trained using the edx training matrix.

Using the optimized parameters and test data a prediction model is made.
The prediction is made with the base prediction of model 7 and the
residuals predicted here. Finally, an RMSE is calculated.

\begin{Shaded}
\begin{Highlighting}[]
\CommentTok{# Use recosystem package to perform matrix factorization}
\KeywordTok{install.packages}\NormalTok{(}\StringTok{"recosystem"}\NormalTok{)}
\KeywordTok{library}\NormalTok{(recosystem)}
\NormalTok{matrix_train_residuals <-}\StringTok{ }\NormalTok{residuals_train_set }\OperatorTok\StringTok{ }\KeywordTok{as.matrix}\NormalTok{() }
\NormalTok{matrix_test <-}\StringTok{ }\NormalTok{edx_test_set }\OperatorTok\StringTok{ }
\StringTok{  }\KeywordTok{select}\NormalTok{(userId, movieId, rating) }\OperatorTok
\StringTok{  }\KeywordTok{as.matrix}\NormalTok{()}

\CommentTok{# write the matrices on disk and assign them to a variable}
\KeywordTok{write.table}\NormalTok{(matrix_train_residuals , }\DataTypeTok{file =} \StringTok{"matrixtrainresiduals.txt"}\NormalTok{ , }\DataTypeTok{sep =} \StringTok{" "}\NormalTok{ , }\DataTypeTok{row.names =} \OtherTok{FALSE}\NormalTok{, }\DataTypeTok{col.names =} \OtherTok{FALSE}\NormalTok{)}
\KeywordTok{write.table}\NormalTok{(matrix_test, }\DataTypeTok{file =} \StringTok{"matrix_test.txt"}\NormalTok{ , }\DataTypeTok{sep =} \StringTok{" "}\NormalTok{ , }\DataTypeTok{row.names =} \OtherTok{FALSE}\NormalTok{, }\DataTypeTok{col.names =} \OtherTok{FALSE}\NormalTok{)}

\KeywordTok{set.seed}\NormalTok{(}\DecValTok{1992}\NormalTok{, }\DataTypeTok{sample.kind =} \StringTok{"Rounding"}\NormalTok{) }
\NormalTok{trainset <-}\StringTok{ }\KeywordTok{data_file}\NormalTok{(}\StringTok{"matrixtrainresiduals.txt"}\NormalTok{)}
\NormalTok{testset <-}\StringTok{ }\KeywordTok{data_file}\NormalTok{(}\StringTok{"matrix_test.txt"}\NormalTok{)}

\CommentTok{# make a recommender object}
\NormalTok{recommender <-}\KeywordTok{Reco}\NormalTok{()}

\CommentTok{# tuning the recommender with the training data}
\NormalTok{opts <-}\StringTok{ }\NormalTok{recommender}\OperatorTok{$}\KeywordTok{tune}\NormalTok{(trainset, }\DataTypeTok{opts =} \KeywordTok{list}\NormalTok{(}\DataTypeTok{dim =} \KeywordTok{c}\NormalTok{(}\DecValTok{10}\NormalTok{, }\DecValTok{20}\NormalTok{, }\DecValTok{30}\NormalTok{), }\DataTypeTok{lrate =} \KeywordTok{c}\NormalTok{(}\FloatTok{0.1}\NormalTok{, }\FloatTok{0.2}\NormalTok{),}
                                      \DataTypeTok{costp_l1 =} \DecValTok{0}\NormalTok{, }\DataTypeTok{costq_l1 =} \DecValTok{0}\NormalTok{,}
                                      \DataTypeTok{nthread =} \DecValTok{1}\NormalTok{, }\DataTypeTok{niter =} \DecValTok{10}\NormalTok{))}
\NormalTok{opts}
\end{Highlighting}
\end{Shaded}

\begin{verbatim}
## $min
## $min$dim
## [1] 30
## 
## $min$costp_l1
## [1] 0
## 
## $min$costp_l2
## [1] 0.01
## 
## $min$costq_l1
## [1] 0
## 
## $min$costq_l2
## [1] 0.1
## 
## $min$lrate
## [1] 0.1
## 
## $min$loss_fun
## [1] 0.8012723
## 
## 
## $res
##    dim costp_l1 costp_l2 costq_l1 costq_l2 lrate  loss_fun
## 1   10        0     0.01        0     0.01   0.1 0.8177840
## 2   20        0     0.01        0     0.01   0.1 0.8283626
## 3   30        0     0.01        0     0.01   0.1 0.8385643
## 4   10        0     0.10        0     0.01   0.1 0.8127029
## 5   20        0     0.10        0     0.01   0.1 0.8133243
## 6   30        0     0.10        0     0.01   0.1 0.8157175
## 7   10        0     0.01        0     0.10   0.1 0.8087027
## 8   20        0     0.01        0     0.10   0.1 0.8038228
## 9   30        0     0.01        0     0.10   0.1 0.8012723
## 10  10        0     0.10        0     0.10   0.1 0.8249721
## 11  20        0     0.10        0     0.10   0.1 0.8234454
## 12  30        0     0.10        0     0.10   0.1 0.8246958
## 13  10        0     0.01        0     0.01   0.2 0.8244819
## 14  20        0     0.01        0     0.01   0.2 0.8407193
## 15  30        0     0.01        0     0.01   0.2 0.8587363
## 16  10        0     0.10        0     0.01   0.2 0.8133382
## 17  20        0     0.10        0     0.01   0.2 0.8201987
## 18  30        0     0.10        0     0.01   0.2 0.8229976
## 19  10        0     0.01        0     0.10   0.2 0.8094115
## 20  20        0     0.01        0     0.10   0.2 0.8086058
## 21  30        0     0.01        0     0.10   0.2 0.8103958
## 22  10        0     0.10        0     0.10   0.2 0.8223475
## 23  20        0     0.10        0     0.10   0.2 0.8216572
## 24  30        0     0.10        0     0.10   0.2 0.8213752
\end{verbatim}

\begin{Shaded}
\begin{Highlighting}[]
\CommentTok{# training the matrix factorization model}

\NormalTok{recommender}\OperatorTok{$}\KeywordTok{train}\NormalTok{(trainset, }\DataTypeTok{opts =} \KeywordTok{c}\NormalTok{(opts}\OperatorTok{$}\NormalTok{min, }\DataTypeTok{nthread =} \DecValTok{1}\NormalTok{, }\DataTypeTok{niter =} \DecValTok{20}\NormalTok{))}
\end{Highlighting}
\end{Shaded}

\begin{verbatim}
## iter      tr_rmse          obj
##    0       0.8595   5.5650e+06
##    1       0.8365   5.1695e+06
##    2       0.8194   5.0296e+06
##    3       0.8025   4.8963e+06
##    4       0.7872   4.7806e+06
##    5       0.7737   4.6833e+06
##    6       0.7620   4.6006e+06
##    7       0.7520   4.5319e+06
##    8       0.7433   4.4752e+06
##    9       0.7357   4.4254e+06
##   10       0.7289   4.3812e+06
##   11       0.7229   4.3452e+06
##   12       0.7176   4.3110e+06
##   13       0.7128   4.2832e+06
##   14       0.7085   4.2564e+06
##   15       0.7047   4.2336e+06
##   16       0.7012   4.2127e+06
##   17       0.6980   4.1945e+06
##   18       0.6951   4.1774e+06
##   19       0.6925   4.1620e+06
\end{verbatim}

\begin{Shaded}
\begin{Highlighting}[]
\CommentTok{# prediction using trained model}

\NormalTok{prediction <-}\StringTok{ }\KeywordTok{tempfile}\NormalTok{()}
\NormalTok{recommender}\OperatorTok{$}\KeywordTok{predict}\NormalTok{(testset, }\KeywordTok{out_file}\NormalTok{(prediction))}
\end{Highlighting}
\end{Shaded}

\begin{verbatim}
## prediction output generated at C:\Users\yanni\AppData\Local\Temp\Rtmpis4QB9\file884481942c6
\end{verbatim}

\begin{Shaded}
\begin{Highlighting}[]
\NormalTok{prediction_resid_matfac <-}\StringTok{ }\KeywordTok{scan}\NormalTok{(prediction)}
\NormalTok{prediction_ratings_matfac <-}\StringTok{ }\NormalTok{predicted_ratings_m7 }\OperatorTok{+}\StringTok{ }\NormalTok{prediction_resid_matfac}
\NormalTok{model_}\DecValTok{8}\NormalTok{_RMSE <-}\StringTok{ }\KeywordTok{RMSE}\NormalTok{(prediction_ratings_matfac, edx_test_set}\OperatorTok{$}\NormalTok{rating)}
\NormalTok{model_}\DecValTok{8}\NormalTok{_RMSE}
\end{Highlighting}
\end{Shaded}

\begin{verbatim}
## [1] 0.7968553
\end{verbatim}

\begin{Shaded}
\begin{Highlighting}[]
\NormalTok{all_RMSEs <-}\StringTok{ }\KeywordTok{bind_rows}\NormalTok{(all_RMSEs,}
                       \KeywordTok{data_frame}\NormalTok{(}\DataTypeTok{method=}\StringTok{"Model 8: Matrix factorization"}\NormalTok{,}
                                  \DataTypeTok{RMSE =}\NormalTok{ model_}\DecValTok{8}\NormalTok{_RMSE))}
\NormalTok{all_RMSEs }\OperatorTok\StringTok{ }\NormalTok{knitr}\OperatorTok{::}\KeywordTok{kable}\NormalTok{()}
\end{Highlighting}
\end{Shaded}

\begin{longtable}[]{@{}lr@{}}
\toprule
method & RMSE\tabularnewline
\midrule
\endhead
Model 1 : overall average & 1.0606351\tabularnewline
Model 2: movie bias & 0.9440709\tabularnewline
Model 3: movie bias + user bias & 0.8664802\tabularnewline
Model 4: regularized movie bias & 0.9440293\tabularnewline
Model 5: regularized user bias & 0.9785199\tabularnewline
Model 6: regularized movie bias + regularized user bias &
0.8658766\tabularnewline
Model 7: regularized movie bias + regularized user bias 2 &
0.8658479\tabularnewline
Model 8: Matrix factorization & 0.7968553\tabularnewline
\bottomrule
\end{longtable}

\hypertarget{final-validation-on-model-8}{%
\subsubsection{Final validation on model
8}\label{final-validation-on-model-8}}

The RMSE of model 8 is the lowest of all models. However, the RMSE was
calculated using the edx test set, which was also used to optimize the
lambda in model 7. It is, however, not done in machine learning to use
(part of) a data set for training and validation! Therefore, the
recommendation algorithm of model 8 will be validated using the
validation set separated from the Movielens 10M data set. This data set
has not been used for training the algorithms and is thus perfect for
validating model 8.

For the baseline model the same steps as for the development of model 7
are used, while for the residuals the matrix factorization method from
model 8 is used. Here the complete edx set is used as a training set and
the lambda derived in model 7 is used as regularization parameter for
the movie and user effect.

\begin{Shaded}
\begin{Highlighting}[]
\CommentTok{## Final validation with model 8}
\CommentTok{# Now use complete edx set for training}
\CommentTok{#first use model 7 for regularized user and movie bias}
\NormalTok{avg_rating <-}\StringTok{ }\KeywordTok{mean}\NormalTok{(edx}\OperatorTok{$}\NormalTok{rating)}
\NormalTok{lambdas <-}\StringTok{ }\KeywordTok{seq}\NormalTok{(}\DecValTok{0}\NormalTok{, }\DecValTok{10}\NormalTok{, }\FloatTok{0.25}\NormalTok{)}

\NormalTok{um_lambda <-}\StringTok{ }\FloatTok{4.75}
\NormalTok{avg_rating <-}\StringTok{ }\KeywordTok{mean}\NormalTok{(edx}\OperatorTok{$}\NormalTok{rating)}

\NormalTok{reg_movie_bias_valid <-}\StringTok{ }\NormalTok{edx }\OperatorTok\StringTok{ }
\StringTok{  }\KeywordTok{group_by}\NormalTok{(movieId) }\OperatorTok\StringTok{ }
\StringTok{  }\KeywordTok{summarize}\NormalTok{(}\DataTypeTok{reg_m_bias =}\NormalTok{ (}\KeywordTok{sum}\NormalTok{(rating }\OperatorTok{-}\StringTok{ }\NormalTok{avg_rating))}\OperatorTok{/}\NormalTok{(}\KeywordTok{n}\NormalTok{()}\OperatorTok{+}\NormalTok{um_lambda))}
\NormalTok{reg_user_bias_valid <-}\StringTok{ }\NormalTok{edx }\OperatorTok\StringTok{ }
\StringTok{  }\KeywordTok{left_join}\NormalTok{(reg_movie_bias_valid, }\DataTypeTok{by=}\StringTok{'movieId'}\NormalTok{) }\OperatorTok
\StringTok{  }\KeywordTok{group_by}\NormalTok{(userId) }\OperatorTok\StringTok{ }
\StringTok{  }\KeywordTok{summarize}\NormalTok{(}\DataTypeTok{reg_u_bias =}\NormalTok{ (}\KeywordTok{sum}\NormalTok{(rating }\OperatorTok{-}\StringTok{ }\NormalTok{avg_rating }\OperatorTok{-}\StringTok{ }\NormalTok{reg_m_bias))}\OperatorTok{/}\NormalTok{(}\KeywordTok{n}\NormalTok{()}\OperatorTok{+}\NormalTok{um_lambda))}

\NormalTok{predicted_ratings_validation <-}\StringTok{ }
\StringTok{  }\NormalTok{validation }\OperatorTok\StringTok{ }
\StringTok{  }\KeywordTok{left_join}\NormalTok{(reg_movie_bias_valid, }\DataTypeTok{by =} \StringTok{"movieId"}\NormalTok{) }\OperatorTok
\StringTok{  }\KeywordTok{left_join}\NormalTok{(reg_user_bias_valid, }\DataTypeTok{by =} \StringTok{"userId"}\NormalTok{) }\OperatorTok
\StringTok{  }\KeywordTok{mutate}\NormalTok{(}\DataTypeTok{pred =}\NormalTok{ avg_rating }\OperatorTok{+}\StringTok{ }\NormalTok{reg_m_bias }\OperatorTok{+}\StringTok{ }\NormalTok{reg_u_bias)}\OperatorTok
\StringTok{  }\KeywordTok{pull}\NormalTok{(pred) }

\NormalTok{residuals_edx_set <-}\StringTok{ }\NormalTok{edx }\OperatorTok\StringTok{ }
\StringTok{  }\KeywordTok{left_join}\NormalTok{(reg_movie_bias_valid, }\DataTypeTok{by =} \StringTok{"movieId"}\NormalTok{) }\OperatorTok
\StringTok{  }\KeywordTok{left_join}\NormalTok{(reg_user_bias_valid, }\DataTypeTok{by =} \StringTok{"userId"}\NormalTok{) }\OperatorTok
\StringTok{  }\KeywordTok{mutate}\NormalTok{(}\DataTypeTok{residual =}\NormalTok{ rating }\OperatorTok{-}\StringTok{ }\NormalTok{avg_rating }\OperatorTok{-}\StringTok{ }\NormalTok{reg_m_bias }\OperatorTok{-}\StringTok{ }\NormalTok{reg_u_bias) }\OperatorTok
\StringTok{  }\KeywordTok{select}\NormalTok{(userId, movieId, residual)}
\KeywordTok{head}\NormalTok{(residuals_edx_set)}
\end{Highlighting}
\end{Shaded}

\begin{verbatim}
##   userId movieId  residual
## 1      1     122 0.7967788
## 2      1     185 0.5273185
## 3      1     292 0.2387459
## 4      1     316 0.3070651
## 5      1     329 0.3192734
## 6      1     355 1.1679938
\end{verbatim}

\begin{Shaded}
\begin{Highlighting}[]
\NormalTok{matrix_edx_residuals <-}\StringTok{ }\NormalTok{residuals_edx_set }\OperatorTok\StringTok{ }\KeywordTok{as.matrix}\NormalTok{() }
\NormalTok{matrix_validation <-}\StringTok{ }\NormalTok{validation }\OperatorTok\StringTok{ }
\StringTok{  }\KeywordTok{select}\NormalTok{(userId, movieId, rating) }\OperatorTok
\StringTok{  }\KeywordTok{as.matrix}\NormalTok{()}

\KeywordTok{write.table}\NormalTok{(matrix_edx_residuals , }\DataTypeTok{file =} \StringTok{"matrixedxresiduals.txt"}\NormalTok{ , }\DataTypeTok{sep =} \StringTok{" "}\NormalTok{ , }\DataTypeTok{row.names =} \OtherTok{FALSE}\NormalTok{, }\DataTypeTok{col.names =} \OtherTok{FALSE}\NormalTok{)}
\KeywordTok{write.table}\NormalTok{(matrix_validation, }\DataTypeTok{file =} \StringTok{"matrix_validation.txt"}\NormalTok{ , }\DataTypeTok{sep =} \StringTok{" "}\NormalTok{ , }\DataTypeTok{row.names =} \OtherTok{FALSE}\NormalTok{, }\DataTypeTok{col.names =} \OtherTok{FALSE}\NormalTok{)}

\KeywordTok{set.seed}\NormalTok{(}\DecValTok{1992}\NormalTok{, }\DataTypeTok{sample.kind =} \StringTok{"Rounding"}\NormalTok{) }
\NormalTok{edxset <-}\StringTok{ }\KeywordTok{data_file}\NormalTok{(}\StringTok{"matrixedxresiduals.txt"}\NormalTok{)}
\NormalTok{validationset <-}\StringTok{ }\KeywordTok{data_file}\NormalTok{(}\StringTok{"matrix_validation.txt"}\NormalTok{)}

\CommentTok{# make a recommender object}
\NormalTok{recommender_valid <-}\KeywordTok{Reco}\NormalTok{()}

\CommentTok{# tuning the recommender with the training data}
\NormalTok{opts_valid <-}\StringTok{ }\NormalTok{recommender_valid}\OperatorTok{$}\KeywordTok{tune}\NormalTok{(edxset, }\DataTypeTok{opts =} \KeywordTok{list}\NormalTok{(}\DataTypeTok{dim =} \KeywordTok{c}\NormalTok{(}\DecValTok{10}\NormalTok{, }\DecValTok{20}\NormalTok{, }\DecValTok{30}\NormalTok{), }\DataTypeTok{lrate =} \KeywordTok{c}\NormalTok{(}\FloatTok{0.1}\NormalTok{, }\FloatTok{0.2}\NormalTok{),}
                                               \DataTypeTok{costp_l1 =} \DecValTok{0}\NormalTok{, }\DataTypeTok{costq_l1 =} \DecValTok{0}\NormalTok{,}
                                               \DataTypeTok{nthread =} \DecValTok{1}\NormalTok{, }\DataTypeTok{niter =} \DecValTok{10}\NormalTok{))}
\NormalTok{opts_valid}
\end{Highlighting}
\end{Shaded}

\begin{verbatim}
## $min
## $min$dim
## [1] 30
## 
## $min$costp_l1
## [1] 0
## 
## $min$costp_l2
## [1] 0.01
## 
## $min$costq_l1
## [1] 0
## 
## $min$costq_l2
## [1] 0.1
## 
## $min$lrate
## [1] 0.1
## 
## $min$loss_fun
## [1] 0.7941609
## 
## 
## $res
##    dim costp_l1 costp_l2 costq_l1 costq_l2 lrate  loss_fun
## 1   10        0     0.01        0     0.01   0.1 0.8072272
## 2   20        0     0.01        0     0.01   0.1 0.8114371
## 3   30        0     0.01        0     0.01   0.1 0.8200044
## 4   10        0     0.10        0     0.01   0.1 0.8049610
## 5   20        0     0.10        0     0.01   0.1 0.8018715
## 6   30        0     0.10        0     0.01   0.1 0.8034186
## 7   10        0     0.01        0     0.10   0.1 0.8038635
## 8   20        0     0.01        0     0.10   0.1 0.7973237
## 9   30        0     0.01        0     0.10   0.1 0.7941609
## 10  10        0     0.10        0     0.10   0.1 0.8242639
## 11  20        0     0.10        0     0.10   0.1 0.8235046
## 12  30        0     0.10        0     0.10   0.1 0.8228670
## 13  10        0     0.01        0     0.01   0.2 0.8106341
## 14  20        0     0.01        0     0.01   0.2 0.8233611
## 15  30        0     0.01        0     0.01   0.2 0.8368961
## 16  10        0     0.10        0     0.01   0.2 0.8062928
## 17  20        0     0.10        0     0.01   0.2 0.8064956
## 18  30        0     0.10        0     0.01   0.2 0.8082813
## 19  10        0     0.01        0     0.10   0.2 0.8034891
## 20  20        0     0.01        0     0.10   0.2 0.7991972
## 21  30        0     0.01        0     0.10   0.2 0.7990707
## 22  10        0     0.10        0     0.10   0.2 0.8227886
## 23  20        0     0.10        0     0.10   0.2 0.8215954
## 24  30        0     0.10        0     0.10   0.2 0.8219777
\end{verbatim}

\begin{Shaded}
\begin{Highlighting}[]
\CommentTok{# training the matrix factorization model}

\NormalTok{recommender_valid}\OperatorTok{$}\KeywordTok{train}\NormalTok{(edxset, }\DataTypeTok{opts =} \KeywordTok{c}\NormalTok{(opts_valid}\OperatorTok{$}\NormalTok{min, }\DataTypeTok{nthread =} \DecValTok{1}\NormalTok{, }\DataTypeTok{niter =} \DecValTok{20}\NormalTok{))}
\end{Highlighting}
\end{Shaded}

\begin{verbatim}
## iter      tr_rmse          obj
##    0       0.8590   6.9701e+06
##    1       0.8343   6.4465e+06
##    2       0.8159   6.2616e+06
##    3       0.7986   6.0910e+06
##    4       0.7840   5.9561e+06
##    5       0.7717   5.8478e+06
##    6       0.7614   5.7569e+06
##    7       0.7528   5.6832e+06
##    8       0.7455   5.6253e+06
##    9       0.7391   5.5727e+06
##   10       0.7335   5.5290e+06
##   11       0.7287   5.4934e+06
##   12       0.7243   5.4604e+06
##   13       0.7204   5.4312e+06
##   14       0.7169   5.4046e+06
##   15       0.7138   5.3828e+06
##   16       0.7109   5.3620e+06
##   17       0.7083   5.3433e+06
##   18       0.7058   5.3264e+06
##   19       0.7037   5.3123e+06
\end{verbatim}

\begin{Shaded}
\begin{Highlighting}[]
\NormalTok{prediction_validation <-}\StringTok{ }\KeywordTok{tempfile}\NormalTok{()}
\NormalTok{recommender_valid}\OperatorTok{$}\KeywordTok{predict}\NormalTok{(validationset, }\KeywordTok{out_file}\NormalTok{(prediction_validation))}
\end{Highlighting}
\end{Shaded}

\begin{verbatim}
## prediction output generated at C:\Users\yanni\AppData\Local\Temp\Rtmpis4QB9\file88446d0429d8
\end{verbatim}

\begin{Shaded}
\begin{Highlighting}[]
\NormalTok{prediction_validation_resid <-}\StringTok{ }\KeywordTok{scan}\NormalTok{(prediction_validation)}
\NormalTok{prediction_validation_ratings <-}\StringTok{ }\NormalTok{predicted_ratings_validation }\OperatorTok{+}\StringTok{ }\NormalTok{prediction_validation_resid}
\NormalTok{RMSE_validation <-}\StringTok{ }\KeywordTok{RMSE}\NormalTok{(prediction_validation_ratings, validation}\OperatorTok{$}\NormalTok{rating)}
\NormalTok{RMSE_validation}
\end{Highlighting}
\end{Shaded}

\begin{verbatim}
## [1] 0.7865076
\end{verbatim}

\hypertarget{conclusion}{%
\subsection{Conclusion}\label{conclusion}}

This cap stone project has shown how a movie recommendation algorithm
can be developed using data exploration and machine learning. This data
science challenge was carried out using part of the classical MovieLens
dataset. RMSE was used as the parameter to evaluate the different
recommendation models. A summary of the RMSEs obtained during the
development of the recommendation model are shown underneath.

\begin{longtable}[]{@{}lr@{}}
\toprule
method & RMSE\tabularnewline
\midrule
\endhead
Model 1 : overall average & 1.0606351\tabularnewline
Model 2: movie bias & 0.9440709\tabularnewline
Model 3: movie bias + user bias & 0.8664802\tabularnewline
Model 4: regularized movie bias & 0.9440293\tabularnewline
Model 5: regularized user bias & 0.9785199\tabularnewline
Model 6: regularized movie bias + regularized user bias &
0.8658766\tabularnewline
Model 7: regularized movie bias + regularized user bias 2 &
0.8658479\tabularnewline
Model 8: Matrix factorization & 0.7968553\tabularnewline
Model 8 Validation & 0.7865076\tabularnewline
\bottomrule
\end{longtable}

The final RMSE table shows that model 8 has by far the lowest RMSE. This
shows that matrix factorization is a very strong technique to discover
patterns between users and movies based on a large and sparse set of
ratings. These patterns could not be derived from purely analyzing the
explicit information in the dataset. The final validation also confirms
the strength of the recommendation algorithm of model 8. The baseline
model could be further optimized by also adding a genre bias, movieage
bias and a release year bias. In addition, using cross validation
instead of a single test set would be more reliable for determining the
regularization parameter.

\end{document}
